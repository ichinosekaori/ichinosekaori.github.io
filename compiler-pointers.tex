\documentclass{beamer}

\title[Compiling FP]{Pointers to General Resources on FP Language Compiler Construction}
\author{Ichinose Kaori}
\date{January 14, 2024}

\usepackage{tikz-cd}
\usepackage{amsmath}
\usepackage{amssymb}

\usetheme{Boadilla}
\usecolortheme{beaver}

\begin{document}

\maketitle

\begin{frame}{Step 0: Learn FP}
  \begin{itemize}
  \item \textit{Practical Common Lisp}
  \item \textit{On Lisp}
  \item \textit{R\textsuperscript{5}RS}
  \item \textit{The Little Schemer}
  \item \textit{Structure and Interpretation of Computer Programs}
  \item \textit{ML for the Working Programmer}
  \item CS3110
  \end{itemize}
\end{frame}

\begin{frame}{TL; DW}
  \begin{itemize}
  \item \url{https://matt.might.net/articles/cps-conversion/}
  \item \url{https://matt.might.net/articles/compiling-scheme-to-c/}
  \item \url{https://github.com/akeep/scheme-to-c/}
  \item \url{http://churchturing.org/y/90-min-scc.pdf}
  \item \url{https://www.youtube.com/watch?v=Bp89aBm9tGU}
  \item \url{https://www.youtube.com/watch?v=M4dwcdK5bxE}
  \item \url{https://gist.github.com/nyuichi/1116686}
  \item \url{http://scheme2006.cs.uchicago.edu/11-ghuloum.pdf}
  \item \href{https://github.com/cisco/ChezScheme/blob/main/IMPLEMENTATION.md}{ChezScheme/IMPLEMENTATION.md}
  \item \url{https://github.com/ichinosekaori/yass/} (possibly later)
  \item \textit{Compiling with Continuations}
  \end{itemize}
\end{frame}

\begin{frame}{Goals}
  \begin{itemize}
    \item To demonstrate compiling a functional programming language (Scheme) to a fairly low-level language (register VM bytecode)
    \item Do it using a simple functional language
    \item Do it using a fairly conventional VM (with the ISA mimicking commercial CPU designs, e.g. aarch64)
  \end{itemize}
\end{frame}

\begin{frame}{Overview of Scheme (1)}
  Primitive forms:
  \begin{itemize}
  \item Variable reference
  \item Quotation
  \item Procedure call
  \item Abstraction
  \item Assignment
  \item Conditional
  \end{itemize}

  Also derived forms programmed in the same language!

  Some data usually not first-class are first-class: continuations, environments. 
\end{frame}

\begin{frame}{Overview of Scheme (2)}
  Primitive data structures:
  \begin{itemize}
  \item The Cons
  \item Vector
  \item (Bytevector)
  \end{itemize}

  Vectors are imperative arrays.

  Data GCed.
\end{frame}

\begin{frame}{Overview of the VM}
  \begin{itemize}
  \item lea
  \item mov
  \item ld, st
  \item ldi
  \item $R_d\gets R_a\operatorname{op}R_b$ (or unary; for arithmetic and logical operations)
  \item jmp
  \item je
  \item int (``hypercalls'')
  \end{itemize}
  where $R$ can be $X$ for 64-bit integers or $D$ for double-precision floats.
  
  VM for avoiding outputing PE/ELF/Mach-O or amd64/aarch64 machine code.
\end{frame}

\begin{frame}{Merits of programming in Scheme}
  \begin{itemize}
  \item small base
  \item extensibility
  \item \url{http://practical-scheme.net/docs/schemersway.html}
  \end{itemize}
\end{frame}

\begin{frame}{Gaps between source and target languages}
  \begin{itemize}
  \item Nowhere to store computation state
  \item No notion of abstractions in the target
  \item Target works on numbers
  \item No memory management in the target
  \item Registers are limited in number
  \end{itemize}
\end{frame}

\begin{frame}[fragile]{Compiler organization}
  \[\begin{tikzcd}
\text{Scheme} \arrow[d]           & \text{Closed} \arrow[r]      & \text{3-address} \arrow[d, "\text{register allocation}" description]      \\
\text{Primitive Scheme} \arrow[d] & \text{Known-adic} \arrow[u, "\text{closure conversion}"]  & \text{Bounded-free} \arrow[d] \\
\text{Scheme form} \arrow[d]      & \text{CPS} \arrow[u]         & \text{VM bytecode}            \\
\text{Unique-names} \arrow[r]     & \text{Assignless} \arrow[u, "\text{CPS transform}" description] &                              
\end{tikzcd}\]

Labeled passes are more traditional passes found in compilers for functional programming languages.
\end{frame}

\begin{frame}{Passes specific to Scheme}
  \begin{itemize}
  \item Macro expansion
  \item Unsplice
  \item Assignment conversion
  \item Variadic function elimination
  \end{itemize}
\end{frame}

\begin{frame}{Notes on continuations}
  \begin{itemize}
  \item They represent ``rest of the computation''
  \item Semantically is a function
  \end{itemize}

  \begin{example}
    The continuation for the $2$ in $2+3$ is $(\cdot+3)$, and it for the $1\times2$ in $(2\times3)+(1\times2)$ is $(6+\cdot)$. (assuming LtR evaluation order)
  \end{example}
\end{frame}

\begin{frame}{Rationale for CPS}
  \begin{itemize}
  \item Explicit continuations for capture
  \item Continuations reified as functions for free
  \item Less code complexity
  \item All calls are in a tail context after CPS --- space for control moved into the closure for the continuation
  \item More optimization opportunities
  %\item We have mutable state so we get delimited continuations as well (cf. \textit{Representing Monads})
  \end{itemize}
  
  \[callcc(k, f)=f(\lambda k'.\lambda y.k(y), k)\]
\end{frame}

\newcommand{\CPS}{\operatorname{CPS}}

\begin{frame}{CPS for the Lambda Calculus}
  \begin{definition}[Lambda terms]
    \[t ::= x \mid \lambda x.t \mid t\ t\]
  \end{definition}
  
  \begin{theorem}
    \[\CPS(x, k)=k\ x\]
    \[\CPS(\lambda x.t, c)=c(\lambda k.\lambda x. \CPS(t, k))\]
    \[\CPS(t_1(t_2), k)=\CPS(t_1,\lambda r_1.\CPS(t_2,\lambda r_2.r_1(k)(r_2)))\]
  \end{theorem}
\end{frame}

\begin{frame}{Generalizing}
  Scheme has more primitives, including quotations, conditionals, and multiple arguments.

  \[\CPS('a,k)=k('a)\]
  \[\begin{aligned}
    &\CPS(\text{if }c\text{ then }a\text{ else }b, k)\\
    =&\CPS(c, \lambda x.(\text{if }x\text{ then }\CPS(a,k)\text{ else }\CPS(b,k)))
  \end{aligned}\]

For generalizing to $n$-ary functions you need to bind all $n$ operands to names, then apply.
\end{frame}

\begin{frame}[fragile]{Hidden code blowup!}
  \[\begin{aligned}
    &\CPS(\text{if }c\text{ then }a\text{ else }b, k)\\
    =&\CPS(c, \lambda x.(\text{if }x\text{ then }\CPS(a,k)\text{ else }\CPS(b,k)))
  \end{aligned}\]

$k$ appears twice --- bind it before continuing!
\end{frame}

\begin{frame}{CPS TL; DR}
  \begin{center}
    Copy from \url{https://matt.might.net/articles/cps-conversion/}.
  \end{center}
  The article has a fully-featured CPS transform implementation for Scheme.
\end{frame}

\begin{frame}{Closure conversion rationale}
  \begin{example}
    Consider the (different) return values of $\lambda x.\lambda y.x+y$.
  \end{example}

    \[\text{func}: \text{code}\times\text{any list}\]
\end{frame}

\newcommand{\FV}{\operatorname{FV}}
\begin{frame}{Finding free variables of Lambda Calculus terms}
  \begin{theorem}
    \[\FV(x)=\{x\}\]
    \[\FV(t_1(t_2))=\FV(t_1)\cup\FV(t_2)\]
    \[\FV(\lambda x.t)=\FV(t)\setminus\{x\}\]
  \end{theorem}

  No extensions to rules necessary for Scheme extensions to the lambda calculus.
\end{frame}

\newcommand{\cloConv}{\operatorname{ccvt}}
\newcommand{\makeClosure}{\operatorname{mkc}}
\begin{frame}{Closure conversion}
  \[
    \begin{aligned}
      &\cloConv(T=\lambda x.t)\\
      =&\makeClosure(\lambda c.\lambda x.\cloConv(t)[\forall s\in\FV(T).s\to\operatorname{cref}(c, s)], \FV(T))
    \end{aligned}
  \]
  \[
    \cloConv(t_1(t_2))=(\lambda s. s (s) \cloConv(t_2))\cloConv(t_1)
  \]
\end{frame}

\begin{frame}{Embedding Scheme data into machine words}
  \begin{itemize}
  \item tagged union (portable)
  \item tagged pointer (used by Chez)
  \item NaN boxing (used by V8)
  \end{itemize}
\end{frame}

\begin{frame}{Tagged pointers}
  \begin{itemize}
  \item A word is 8-bytes long
  \item Pointers to 8-byte-aligned \textit{things} will have 000 as their LSBs
  \item Use different values of the 3 LSBs to differentiate between types
  \end{itemize}

  See \href{https://github.com/cisco/ChezScheme/blob/main/IMPLEMENTATION.md}{ChezScheme/IMPLEMENTATION.md}.
\end{frame}

\begin{frame}{Managing memory}
  \begin{itemize}
  \item mark-sweep
  \item mark-compact
  \item mark-copy
  \item generational?
  \item concurrent?
  \item parallel?
  \end{itemize}

  Start simple: use Cheney's semispace algorithm
\end{frame}

\begin{frame}{Register allocation}
  \begin{itemize}
  \item Best: do whole-program RA and do coalescing (since control flow is broken up into slices after the CPS pass)
  \item Worse: \textit{whatever correct}.
  \end{itemize}
\end{frame}

\begin{frame}{Ideas for more work}
  \begin{itemize}
  \item More refined types
  \item Evaluation and environments
  \item Light processes
  \item Pattern matching
  \item Multi-dispatch methods
  \item Staged and safe code
  \item FBIP
  \item Zombie!
  \item Native backend
  \item More advanced RA/GC/optimizations
  \end{itemize}
\end{frame}

\begin{frame}
  \begin{center}
    Slides at \url{https://ichinosekaori.github.io/compiler-pointers.pdf}
  \end{center}
\end{frame}

\end{document}
