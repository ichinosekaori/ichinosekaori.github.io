\documentclass{beamer}

\title[FP Compilers]{Pointers to General Resources on FP Language Compiler Construction}
\author{Ichinose Kaori}
\date{January 28, 2024}

\usepackage{tikz-cd}
\usepackage{amsmath}
\usepackage{amssymb}

\usetheme{Boadilla}
\usecolortheme{beaver}

\begin{document}

\maketitle

\begin{frame}{TL; DW (1)}
  \begin{itemize}
  \item \url{https://matt.might.net/articles/cps-conversion/}
  \item \url{https://matt.might.net/articles/compiling-scheme-to-c/}
  \item \url{https://github.com/akeep/scheme-to-c/}
  \item \url{http://churchturing.org/y/90-min-scc.pdf}
  \item \url{https://www.youtube.com/watch?v=Bp89aBm9tGU}
  \item \url{https://www.youtube.com/watch?v=M4dwcdK5bxE}
  \item \url{https://gist.github.com/nyuichi/1116686}
  \item \url{http://scheme2006.cs.uchicago.edu/11-ghuloum.pdf}
  \item \url{https://github.com/ichinosekaori/yass/} (possibly later)
  \item \textit{Compiling with Continuations}
  \end{itemize}
\end{frame}

\begin{frame}{Goals}
  \begin{itemize}
    \item To demonstrate compiling a functional programming language to a fairly low-level language (register VM bytecode)
    \item Do it using a simple functional language
    \item Do it using a fairly conventional VM (with the ISA mimicking commercial CPU designs)
  \end{itemize}
\end{frame}

\begin{frame}{Overview of Scheme (1)}
  Primitive forms:
  \begin{itemize}
  \item Variable reference
  \item Quotation
  \item Procedure call
  \item Abstraction
  \item Assignment
  \item Conditional
  \end{itemize}

  Other forms are built upon these primitives through the use of macros written in the same language (Scheme)

  Some data usually not first-class are first-class: functions, continuations, environments. 
\end{frame}

\begin{frame}{Overview of Scheme (2)}
  Primitive data structures:
  \begin{itemize}
  \item The Cons
  \item Vector
  \item (Bytevector)
  \end{itemize}

  You can build (functional even!) arrays out of conses (think segment trees) but vectors are faster; bytevectors are more compact.

  Data could be freed when no longer accessible, by the garbage collector.
\end{frame}

\begin{frame}{Overview of the VM}
  \begin{itemize}
  \item mov
  \item ld, st
  \item ldi
  \item $R_d\gets R_a\operatorname{op}R_b$ (or unary; for arithmetic and logical operations)
  \item jmp
  \item je
  \item int (``hypercalls'')
  \item Used for avoiding outputing PE/ELF/Mach-O and complex instruction encoding
  \end{itemize}

  where $R$ can be $X$ for 64-bit integers or $D$ for double-precision floats.
\end{frame}

\begin{frame}{Merits of programming in Scheme}
  \begin{itemize}
  \item small base
  \item extensibility
  \item \url{http://practical-scheme.net/docs/schemersway.html}
  \end{itemize}
\end{frame}

\begin{frame}{Gaps between source and target languages}
  \begin{itemize}
  \item No notion of abstractions in the target
  \item No memory management in the target
  \item Target works on numbers
  \item Registers are limited in number
  \item Nowhere to store computation state
  \end{itemize}
\end{frame}

\begin{frame}[fragile]{Compiler organization}
  \[\begin{tikzcd}
\text{Scheme} \arrow[d]           & \text{Closed} \arrow[r]      & \text{Indexed} \arrow[d, "\text{register allocation}" description]      \\
\text{Primitive Scheme} \arrow[d] & \text{Known-adic} \arrow[u, "\text{closure conversion}"]  & \text{Bounded-free} \arrow[d] \\
\text{Scheme form} \arrow[d]      & \text{CPS} \arrow[u]         & \text{VM bytecode}            \\
\text{Unique-names} \arrow[r]     & \text{Assignless} \arrow[u, "\text{CPS transform}" description] &                              
\end{tikzcd}\]

Labeled passes are more traditional passes found in compilers for functional programming languages.
\end{frame}

\begin{frame}{Rationale for CPS}
  \begin{itemize}
  \item Explicit continuations
  \item Continuations reified as functions for free
  %\item We have mutable state so we get delimited continuations as well (cf. \textit{Representing Monads})
  \end{itemize}
  
  \[callcc(k, f)=f(k, \lambda x.\lambda y.k(y))\]
\end{frame}

\newcommand{\CPS}{\operatorname{CPS}}

\begin{frame}{CPS for the Lambda Calculus}
  \[t ::= x \mid \lambda x.t \mid t(t)\]
  
  \[\CPS(x, k)=k(x)\]
  \[\CPS(\lambda x.t, c)=c(\lambda k.\lambda x. \CPS(t, k))\]
  \[\CPS(t_1(t_2), k)=\CPS(t_1,\lambda r_1.\CPS(t_2,\lambda r_2.r_1(k)(r_2)))\]
\end{frame}

\begin{frame}{Generalizing}
  Scheme has more primitives, including quotations, conditionals, and multiple arguments.

  \[\CPS('a,k)=k('a)\]
  \[\begin{aligned}
    &\CPS(\text{if }c\text{ then }a\text{ else }b, k)\\
    =&\CPS(c, \lambda x.(\text{if }x\text{ then }\CPS(a,k)\text{ else }\CPS(b,k)))
  \end{aligned}\]

For generalizing to $n$-ary functions you need to bind all $n$ operands to names, then apply.
\end{frame}

\begin{frame}[fragile]{Hidden code blowup!}
  \[\begin{aligned}
    &\CPS(\text{if }c\text{ then }a\text{ else }b, k)\\
    =&\CPS(c, \lambda x.(\text{if }x\text{ then }\CPS(a,k)\text{ else }\CPS(b,k)))
  \end{aligned}\]

$k$ appears twice --- bind it before continuing!
\end{frame}

\begin{frame}{CPS TL; DR}
  \begin{center}
    Copy from \url{https://matt.might.net/articles/cps-conversion/}.
  \end{center}
  The article has a fully-featured CPS transform implementation for Scheme.
\end{frame}

\begin{frame}{Closure conversion rationale}
  Consider the return values of $\lambda x.\lambda y.x+y$.

  \[\text{func}: \text{code}\times\text{any list}\]
\end{frame}

\newcommand{\FV}{\operatorname{FV}}
\begin{frame}{Finding free variables of Lambda Calculus terms}
  \[\FV(x)=\{x\}\]
  \[\FV(t_1(t_2))=\FV(t_1)\cup\FV(t_2)\]
  \[\FV(\lambda x.t)=\FV(t)\setminus\{x\}\]

  No extensions to rules necessary for Scheme extensions to the lambda calculus.
\end{frame}

\newcommand{\cloConv}{\operatorname{cloConv}}
\newcommand{\makeClosure}{\operatorname{makeClosure}}
\begin{frame}{Closure conversion}
  \[
    \begin{aligned}
      &\cloConv(T=\lambda x.t)\\
      =&\makeClosure(\lambda c.\lambda x.\cloConv(t)[\forall s\in\FV(T).s\to\operatorname{cloRef}(c, s)], \FV(T))
    \end{aligned}
  \]
  \[
    \cloConv(t_1(t_2))=(\lambda s. s (s) \cloConv(t_2))\cloConv(t_1)
  \]
\end{frame}

\begin{frame}{Embedding Scheme data into a machine word}
  \begin{itemize}
  \item tagged union (portable)
  \item tagged pointer (used by Chez)
  \item NaN boxing (used by V8)
  \end{itemize}
\end{frame}

\begin{frame}{Tagged pointers}
  \begin{itemize}
  \item A word is 8-bytes long
  \item Pointers to 8-byte-aligned \textit{things} will have 000 as their LSBs
  \item Use different values of the 3 LSBs to differentiate between types
  \end{itemize}
\end{frame}

\begin{frame}{Managing memory}
  \begin{itemize}
  \item mark-sweep
  \item mark-compact
  \item mark-copy
  \item generational?
  \item concurrent?
  \item parallel?
  \end{itemize}

  Start simple: use Cheney's semispace algorithm
\end{frame}

\begin{frame}{Register allocation}
  \begin{itemize}
  \item Best: do whole-program RA and do coalescing (since control flow is broken up into slices after the CPS pass)
  \item Worse: \textit{whatever correct}
  \end{itemize}
\end{frame}

\end{document}